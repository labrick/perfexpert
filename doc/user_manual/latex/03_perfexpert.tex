\chapter{Using PerfExpert}

The objective of this chapter is to explain how to run programs using PerfExpert and how to interpret its output using a simple matrix multiplication program. In this chapter, we will use the OpenMP simple matrix multiplication program\footnote{\url{https://computing.llnl.gov/tutorials/openMP/samples/C/omp\_mm.c}}. This program multiplies two matrices and prints one value from the resulting matrix.

\nicebox{CAUTION:}{PerfExpert may, if you choose to use the full capabilities for automated optimization, change your source code during the process of optimization. PerfExpert  always saves the original file with a different name (\textit{e.g.}, \texttt{omp\_mm.c.old\_27301}) as well as adding annotations to your source code for each optimization it makes. We cannot, however, fully guarantee that code modifications for optimizations will not break your code. We recommend having a full backup of your original source code before using PerfExpert.}

\section{Environment Configuration}

If you are using any of the TACC\footnote{\url{https://www.tacc.utexas.edu/}} machines, load the appropriate modules:

\begin{verbatim}
$ module load papi hpctoolkit perfexpert
\end{verbatim}

The runs with PerfExpert should be made using a data set size for each compute node which is equivalent to full production runs but for which execution time is not more than about ten or fifteen minutes since PerfExpert will run your application multiple times (actually, three times on Stampede) with different performance counters enabled. For that reason, before you run PerfExpert you should either request iterative access to computational resources (compute node), or modify the job script that you use to run your application and specify a running time that is about 3 (for Stampede) or 6 (for Lonestar) times the normal running time of the program.

To request iterative access to a compute node on Stampede, please, have a look on the User Guide\footnote{\url{https://portal.tacc.utexas.edu/group/tup/user-guides/stampede\#running}}.

Below is an example of a job script file modified to use PerfExpert which runs PerfExpert on the application named my\_program and generate the performance analysis report. Adding command line options will cause suggestions for bottleneck remediation to be generated and output and/or automatic performance optimization to be attempted. 

\pagebreak
\noindent\hrulefill
\begin{verbatim}
#!/bin/bash
#SBATCH -J myMPI             # job name
#SBATCH -o myMPI.o%j         # output and error filename (%j stands to jobID)
#SBATCH -n 16                # total number of mpi tasks requested
#SBATCH -p development       # queue (partition) -- normal, development, etc.
#SBATCH -t 01:30:00          # run time (hh:mm:ss) - 1.5 hours
perfexpert 0.1 ./my_program  # run the executable named my_program\end{verbatim}\hrulefill
	
\section{PerfExpert Options}

There are several different options for applying PerfExpert. The following summary shows you how to choose the options to run PerfExpert to match your needs.

\begin{verbatim}
$ perfexpert -h
Usage: perfexpert <threshold> [-m target|-s sourcefile] [-r count] [-d database]
                  [-p prefix] [-b filename] [-a filename] [-l level] [-gvch]
                  [-k card [-P prefix] [-B filename] [-A filename] ]
                  <program_executable> [program_arguments]

  <threshold>        Define the relevance (in % of runtime) of code fragments PerfExpert
                     should take into consideration (> 0 and <= 1)
  -m --makefile      Use GNU standard `make' command to compile the code (it requires
                     the source code available in current directory)
  -s --source        Specify the source code file (if your source code has more than one
                     file please use a Makefile and choose `-m' option it also enables
                     the automatic optimization option `-a')
  -r --recommend     Number of recommendations (`count') PerfExpert should show
  -d --database      Select the recommendation database file
                     (default: PERFEXPERT_VARDIR/RECOMMENDATION_DB)
  -p --prefix        Add a prefix to the command line (e.g. mpirun). Use double quotes to
                     specify arguments with spaces within (e.g. -p "mpirun -n 2"). Use a
                     semicolon (`;') to run multiple commands in the same command line
  -b --before        Execute 'filename' before each run of the application
  -a --after         Execute 'filename' after each run of the application
  -k --knc           Tell PerfExpert to run the experiments on the KNC 'card'
  -p --prefix-knc    Add a prefix to the command line (e.g. mpirun). Use double quotes to
                     specify arguments with spaces within (e.g. -p "mpirun -n 2"). Use a
                     semicolon (`;') to run multiple commands in the same command line
  -B --knc-before    Execute 'filename' before each run of the application on the KNC
  -A --knc-after     Execute 'filename' after each run of the application on the KNC
  -g --clean-garbage Remove temporary files after run
  -v --verbose       Enable verbose mode using default verbose level (1)
  -l --verbose-level Enable verbose mode using a specific verbose level (1-10)
  -c --colorful      Enable colors on verbose mode, no weird characters will appear on
                     output files
  -h --help          Show this message

Use CC, CFLAGS and LDFLAGS to select compiler and compilation/linkage flags
\end{verbatim}

If you select the \texttt{-m} or \texttt{-s} options, PerfExpert will try to automatically optimize your code and show you the performance analysis report \& the list of suggestion for bottleneck remediation when no automatic optimization is possible.

For the \texttt{-m} or \texttt{-s} options, PerfExpert requires access to the application source code. If you select the \texttt{-m} option and the application is composed of multiple files, your source code tree should have a \texttt{Makefile} file to enable PerfExpert compile your code. If your application is composed of a single source code file, the option \texttt{-s} is sufficient for you. If you do not select \texttt{-m} or \texttt{-s} options, PerfExpert requires only the binary code and will show you only the performance analysis report and the list of suggestion for bottleneck remediation.

PerfExpert will run your application multiple times to collect different performance metrics. You may use the \texttt{-b} (or \texttt{-a}) options if you want to execute a program or script before (or after) each run. The argument \texttt{program\_executable} should be the filename of the application you want to analyze, not a shell script, otherwise, PerfExpert will analyze the performance of the shell script instead of the performance of you application.

Use the \texttt{-r} option to select the number of recommendations for optimization you want for each code section which is a performance bottleneck.

\nicebox{CAUTION:}{If your program takes any argument that starts with a ``\texttt{-}'' signal PerfExpert will interpret this as a command line option. To help PerfExpert handle \texttt{program\_arguments} correctly, use quotes and add a space before the program's arguments (e.g., ``\texttt{ -s 50}'').}

\nicebox{CAUTION:}{In case you are trying to optimize a MPI application, you should use the \texttt{-p} option to specify the MPI launcher and also it's arguments.}

For this guide, using the OpenMP simple matrix multiply code, we will use the following command line options:

\begin{verbatim}
$ OMP_NUM_THREADS=16 CFLAGS="-fopenmp" perfexpert -s mm_omp.c 0.05 mm_omp
\end{verbatim}

which executes PerfExpert's automatic optimizations and will also generate an OpenMP-enabled binary which will run with 16 threads. In this case, PerfExpert will compile the \texttt{mm\_omp.c} code using the system's default compiler, which is GCC in the case of Stampede. PerfExpert will take into consideration only code fragments (loops and functions) that take more 5\% of the runtime.

To select a different compiler, you should specify the \texttt{CC} environment variable as below:

\begin{verbatim}
$ CC="icc" OMP_NUM_THREADS=16 CFLAGS="-fopenmp" perfexpert -s mm_omp.c 0.05 mm_omp
\end{verbatim}

If you do not want to have PerfExpert trying to optimize the application automatically, just compile your program and run the following command:

\begin{verbatim}
$ OMP_NUM_THREADS=16 perfexpert 0.05 mm_omp
\end{verbatim}

\nicebox{WARNING:}{If the command line you use to run PerfExpert includes the MPI launcher (\textit{i.e.}, \texttt{mpirun -n 2 my\_mpi\_app my\_mpi\_app\_arguments ...}), PerfExpert will analyze the performance of the MPI launcher instead of the performance of your application. Use the \texttt{-p} command line argument of PerfExpert to set the MPI launcher and all its arguments (\textit{e.g.}, \texttt{-p "mpirun -n 16"}).}

\section{The Performance Analysis Report}

This section explains the performance analysis report and the metrics shown by PerfExpert. We discuss the following sample output:

\begin{verbatim}
Loop in function compute() (99.9% of the total runtime)
=========================================================================
ratio to total instrns       %  0.......25.........50.......75......100
   - floating point      :    6 ***
   - data accesses       :   33 ****************

* GFLOPS (% max)         :    7 ***
   - packed              :    0 *
   - scalar              :    7 ***
-----------------------------------------------------------------------
performance assessment     LCPI good....okay....fair....poor....bad...
* overall                :  0.8 >>>>>>>>>>>>>>>>
upper bound estimates
* data accesses          :  2.5 >>>>>>>>>>>>>>>>>>>>>>>>>>>>>>>>>>>>>>+
   - L1d hits            :  1.3 >>>>>>>>>>>>>>>>>>>>>>>>>
   - L2d hits            :  0.3 >>>>>
   - L3d hits            :  0.0 >>>>>>>>>>>>>>>>
   - LLC misses          :  0.1 >>
* instruction accesses   :  0.3 >>>>>>
   - L1i hits            :  0.3 >>>>>>
   - L2i hits            :  0.0 >
   - L2i misses          :  0.0 >
* data TLB               :  1.5 >>>>>>>>>>>>>>>>>>>>>>>>>>>>>>>
* instruction TLB        :  0.0 >
* branch instructions    :  0.0 >
   - correctly predicted :  0.0 >
   - mispredicted        :  0.0 >
* floating-point instr   :  0.2 >>>>
   - fast FP instr       :  0.2 >>>>
   - slow FP instr       :  0.0 >
\end{verbatim}

Apart from the total running time, PerfExpert performance analysis report includes, for each code segment:

\begin{itemize}
	\item Instruction execution ratios (with respect to total instructions);
	\item Approximate information about the computational efficiency (GFLOPs measurements);
	\item Overall performance;
	\item Local Cycles Per Instruction (LCPI) values for the cost of memory accesses.
\end{itemize}

The program composition part shows what percentage of the total instructions were computational (floating-point instructions) and what percentage were instructions that accessed data. This gives a rough estimate in trying to understand whether optimizing the program for either data accesses or floating-point instructions would have a significant impact on the total running time of the program.

The PerfExpert performance analysis report also shows the GFLOPs rating, which is the number of floating-point operations executed per second in multiples of 109. The value for this metric is displayed as a percentage of the maximum possible GFLOP value for that particular machine. Although it is rare for real-world programs to match even 50\% of the maximum value, this metric can serve as an estimate of how efficiently the code performs computations.

The next, and major, section of the PerfExpert performance analysis report shows the LCPI values, which is the ratio of cycles spent in the code segment for a specific category, divided by the total number of instructions in the code segment. The overall value is the ratio of the total cycles taken by the code segment to the total instructions executed in the code segment. 

Generally, a value of 0.5 or lower for an LCPI is considered to be good. However, it is only necessary to look at the ratings (\texttt{good}, \texttt{okay}, \ldots, \texttt{bad}) The rest of the report maps this overall LCPI, into the six constituent categories: data accesses, instruction accesses, data TLB accesses, instruction TLB accesses, branches and floating point computations. Without getting into the details of instruction operation on Intel and AMD chips, one can say that these six categories record performance in non-overlapping ways. That is, they roughly represent six separate categories of performance for any application.

The LCPI value is a good indicator of the cost arising from instructions of the specific category. Hence, the higher the LCPI, the slower the program. The following is a brief description of each of these categories:

\begin{description}
	\item[Data accesses:]\hfill \\
	counts the LCPI arising from accesses to memory for program variables.
	\item[Instruction accesses:]\hfill \\
	counts the LCPI arising from memory accesses for code (functions and loops).
	\item[Data TLB:]\hfill \\
	provides an approximate measure of penalty arising from strides in accesses or regularity of accesses.
	\item[Instruction TLB:]\hfill \\
	reflects cost of fetching instructions due to irregular accesses.
	\item[Branch instructions:]\hfill \\
	counts cost of jumps (i.e. if statements, loop conditions, etc.).
	\item[Floating-point instructions:]\hfill \\
	counts LCPI from executing computational (floating-point) instructions.
\end{description}

Some of these LCPI categories have subcategories. For instance, the LCPI from data and instruction accesses can be divided into LCPI arising from the individual levels of the data and instruction caches and branch LCPIs can be divided into LCPIs from correctly predicted and from mispredicted branch instructions. For floating-point instructions, the division is based on floating-point instructions that take few cycles to execute (\textit{e.g.}, add, subtract and multiply instructions) and on floating-point instructions that take longer to execute (\textit{e.g.}, divide and square-root instructions).

In each case, the classification (data access, instruction access, data TLB, etc.) is shown so that it is easy to understand which category is responsible for the performance slowdown. For instance if the overall CPI is ``\texttt{poor}'' and the data access LCPI is high, then you should concentrate on access to program variables and memory. Additional LCPI details help in relating performance numbers to the process architecture.

\nicebox{IMPORTANT:}{When PerfExpert runs with automatic performance optimization enabled the performance analysis report shown reflects the performance of the code after all possible automatic optimizations have been applied.}

\nicebox{NOTICE:}{PerfExpert creates a \texttt{.perfexpert-temp.XXXXXX} directory for each time it is executed. This directory has one subdirectory for each optimization cycle PerfExpert completed or attempted. Each subdirectory includes the intermediate files PerfExpert generated during each cycle, including the performance analysis reports.}

\section{List of Recommendations for Optimization}

If PerfExpert runs with \texttt{-�r} option enabled, it will also generate a list of suggestions for performance improvement for each bottleneck. This option is always available, it does not depend on which of the other command line option are. A list of suggestions for this example is shown below:

\begin{verbatim}
#--------------------------------------------------
# Recommendations for /work/02204/fialho/tutorial/3/mm_omp.c:14
#--------------------------------------------------
#
# Here is a possible recommendation for this code segment
#
Description: move loop invariant computations out of loop
Reason: this optimization reduces the number of executed floating-point operations
Code example:
loop i
  x = x + a * b * c[i];
 =====>
temp = a * b;
loop i
  x = x + temp * c[i];

#
# Here is a possible recommendation for this code segment
#
Description: change the order of loops
Reason: this optimization may improve the memory access pattern and make it more cache 
and TLB friendly
Code example:
loop i
  loop j {...}
 =====>
loop j
  loop i {...}

#
# Here is a possible recommendation for this code segment
#
Description: componentize important loops by factoring them into their own subroutines
Reason: this optimization may allow the compiler to optimize the loop independently and
thus tune it better
Code example:
loop i {...}
...
loop j {...}
 =====>
void li() { loop i {...} }
void lj() { loop j {...} }
...
li();
...
lj();
\end{verbatim}
